\documentclass[10.5pt
%,draft
]{article}

\usepackage{CTEX}
\usepackage{graphicx}
\usepackage{amsmath}
\usepackage{natbib}

\begin{document}

\renewcommand{\refname}{参考文献}
\renewcommand{\figurename}{图}
\renewcommand{\abstractname}{摘要}

\title{行波方程和 Burgers 方程的数值计算--第2次作业\footnote{2019年秋季《磁流体力学的数值模拟方法》}}

\author{郑惠南$^{\footnote{Email:
hue@ustc.edu.cn}}$\footnotetext[7]{Email: hue@ustc.edu.cn}}

\date{%
\scriptsize%
%CAS Key Laboratory for Basic Plasma Physics, School of Earth and Space Sciences,
%\\
%University of Science and Technology of China, Hefei, Anhui 230026, China
中国科学技术大学地球与空间科学学院, 合肥 230026
%
}

\maketitle

\begin{abstract}
讨论一维单一变量双曲型方程(行波方程和Burgers方程)的有限差分数值解法,
结合理论分析讨论各种常用差分格式的特点. 进一步规范作业要求,
请在\textbf{2019年10月16日周三0:00}前完成并提交.
\end{abstract}

\section{引言}
一维单变量函数的一阶双曲型偏微分方程表示了一个波动的传播发展过程, 在线性方程的情况下,
波的形状不随时间改变, 而在非线性方程的情况下, 一个连续有限振幅的波可能发展成有间断的解, 即激波, 或者从激波 (间断) 演化为连续的解\citep{Whitham1999}.
通过偏微分方程的数值计算格式, 可以数值分析解的物理特性和讨论格式的可靠性.

\section{行波方程}
考察方程
\begin{align}
& \frac{\partial u}{\partial t} + \frac{\partial u}{\partial x} = 0,
\label{EqnCon}
\end{align}
在初值条件
\begin{align}
& u|_{t=0} = \left\{\begin{array}{ll} 0.0, & x < -0.4, \\
1.0 - |x + 0.3| / 0.1, & -0.4 \le x < -0.2, \\
0.0, & -0.2 \le x < -0.1, \\
1.0 , & -0.1 \le x < 0.0, \\
0.0, & x \ge 0.0
\end{array}\right.
\end{align}
下的数值解. 通过有限差分格式, 如迎风格式
\begin{align}
u_j^{n+1} = u^n - \frac{\Delta t}{\Delta x} (u_j^n - u_{j-1}^n), \label{EqnUpwind}
\end{align}
讨论该方程的物理解和差分格式数值解的特性. 要求
\begin{enumerate}
\item 设计和编写程序进行数值计算试验, 试验和比较不同格式, 不同网格密度, 不同时间步长的计算结果.
\item 以图形来表示数值解的发展和变化. 和解析解比较, 分析数值解在间断附近的特性, 如速度, 过渡区宽度等.
\item 试验一下不稳定差分格式 (如空间前向差分的Euler格式) 的数值计算过程, 说明所得到的结果.
\item 如果可能, 分析数值耗散和色散效应.
\end{enumerate}

作为参考, 图\ref{LinearW}是方程(\ref{LinearW})的不同格式, 不同Courant系数 ($C = \Delta t / \Delta x$), 在 $t = 0.5$ 的计算结果. 其中, 实线是精确解, 点线表示初态值, 虚线是数值计算结果, 而圆圈是具体网格上的数据. 
\begin{figure}
\begin{center}
\begin{tabular}{cc}
\includegraphics[width=.45\textwidth]{LinearW517005}
&
\includegraphics[width=.45\textwidth]{LinearW517050}
\\[-10pt]
(a) & (b)
\\[8pt]
\includegraphics[width=.45\textwidth]{LinearW517095}
&
\includegraphics[width=.45\textwidth]{LinearW517100}
\\[-10pt]
(c) & (d)
\\[8pt]
\includegraphics[width=.45\textwidth]{LinearWLW517095}
&
\includegraphics[width=.45\textwidth]{LinearWMM517095}
\\[-10pt]
(e) & (f)
\end{tabular}
\end{center}
\caption{方程(\ref{EqnCon})在$t=0.5$时刻的解. 其中虚线表示数值迎风格式的计算结果, 线上的圆圈表示具体网格上的数据. 同一时刻的精确解用实线表示. 作为对照, 初始时刻的值以点线表示, 坐标网格总数为517. (a) 迎风格式, $C = 0.05$, (b)  迎风格式, $C = 0.50$, (c)  迎风格式, $C = 0.95$, (d) 迎风格式, $C = 1.0$, 此时数值结合精确解完全符合, (e) Lax-Wendroff格式, $C=0.95$, 注意其中出现的色散 (上冲和下冲), (f) Minmod格式, $C=0.95$.} \label{LinearW}
\end{figure}

\section{Burgers方程}
考察方程
\begin{align}
\frac{\partial u}{\partial t} + u \frac{\partial u}{\partial x} = 0, \label{EqnBurgers}
\end{align}
在初值为
\begin{align}
u|_{t=0} = \left\{\begin{array}{ll} 1.8, & x < -0.8,
\\
1.4 + 0.4 \cos\left[2 \pi (x + 0.8) \right], & -0.8 \le x < -0.3,
\\
1.0, & -0.3 \le x < 0.0,
\\
1.8, & x \ge 0.0
\end{array} \right.
\end{align}
时的数值解. 设计一到多个差分格式, 分析数值计算结果并进行讨论. 要求
\begin{enumerate}
\item 设计和编写程序进行数值计算试验.
\item 以图形来表示数值解的发展和变化.
\item 可能的话, 分析数值耗散和色散效应 (间断的宽度, 传播速度, 产生的数值解振荡等).
\end{enumerate}

图\ref{BurgersW}是方程(\ref{EqnBurgers})的迎风格式在时间$t = 0.25$, $0.5$, $0.75$, 和$1.0$的计算结果, 其中Courant系数取0.95, 即$\Delta t = 0.95 \frac{\Delta x}{\max(u)}$.
\begin{figure}
\begin{center}
\begin{tabular}{cc}
\includegraphics[width=.45\textwidth]{Burgers025}
&
\includegraphics[width=.45\textwidth]{Burgers050}
\\[-10pt]
(a) & (b)
\\
\includegraphics[width=.45\textwidth]{Burgers075}
&
\includegraphics[width=.45\textwidth]{Burgers100}
\\[-10pt]
(c) & (d)
\end{tabular}
\end{center}
\caption{%
方程(\ref{EqnBurgers})的Minmode格式计算结果, 采用了517个网格, Courant系数$C$取为0.95. 虚线表示数值格式的计算结果, 其中上面的圆圈表示具体的数据; 解析解用实线表示, 初始时刻的值以点线表示. (a) $t=0.25$, (b) $t=0.5$, (c) $t=0.75$, (d) $t=1.0$.
}
\label{BurgersW}
\end{figure}

\section*{附件}

\begin{enumerate}
\item
assign2.tex--本报告的\LaTeX 文件
\item
assign2.pdf--本报告的PDF输出文件
\item
References.bib - 文献文件
\item
LinearW517005.eps - 行波方程(\ref{EqnCon})的数值计算结果, 对应图\ref{LinearW}(a)
\item
LinearW517050.eps - 行波方程(\ref{EqnCon})的数值计算结果, 对应图\ref{LinearW}(b)
\item
LinearW517095.eps - 行波方程(\ref{EqnCon})的数值计算结果, 对应图\ref{LinearW}(c)
\item
LinearW517100.eps - 行波方程(\ref{EqnCon})的数值计算结果, 对应图\ref{LinearW}(d)
\item
LinearWLW517095.eps - 行波方程(\ref{EqnCon})的数值计算结果, 对应图\ref{LinearW}(c)
\item
LinearWMM517095.eps - 行波方程(\ref{EqnCon})的数值计算结果, 对应图\ref{LinearW}(d)
\item
Burgers010.eps - Burgers方程(\ref{EqnBurgers})的数值计算结果, 对应图\ref{BurgersW}(a)
\item
Burgers025.eps - Burgers方程(\ref{EqnBurgers})的数值计算结果, 对应图\ref{BurgersW}(b)
\item
Burgers100.eps - Burgers方程(\ref{EqnBurgers})的数值计算结果, 对应图\ref{BurgersW}(c)
\item
Burgers400.eps - Burgers方程(\ref{EqnBurgers})的数值计算结果, 对应图\ref{BurgersW}(d)
\end{enumerate}

\section*{项目作业的一些要点}
为了更好地帮助同学们完成项目作业, 现将一些注意的检查要点和评分标准 (括弧中的数字) 罗列如下,
\begin{itemize}
\item
期限(20)
\item
完整性/报告/清单/分工(10)
\item
报告源文件和PDF输出文件, 源程序(10)
\item
图形文件(10): EPS或PDF格式文件
\item
规范-语法(10): 论文的写作规范, 如文献引用, 图/表, 叙述之间的逻辑关联, 等等.
\item
变量公式规范(10)
\item
正确性(10)
\item
其它(5): 结构, 概念, 表达, 排版, 等等
\end{itemize}

\bibliographystyle{apalike}
\bibliography{References}

\end{document}
